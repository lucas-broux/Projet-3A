\documentclass[11pt,letterpaper]{article}
\usepackage[utf8]{inputenc}
\usepackage[french]{babel}
\usepackage{amsmath}
\usepackage{amsfonts}
\usepackage{amssymb}
\usepackage{graphicx}
\usepackage{amsthm}
\usepackage{url}
\usepackage[left=3cm,right=3cm,top=3cm,bottom=3cm]{geometry}
\usepackage{makeidx}
\makeatletter
\newcommand{\mathleft}{\@fleqntrue\@mathmargin0pt}
\newcommand{\mathcenter}{\@fleqnfalse}
\makeatother
\setlength\parindent{0pt}
\makeindex

% Define 2 graphic paths to images.
\graphicspath{{images/}{../images/}}

% Import subfiles package. 
\usepackage{subfiles}


\title{Rapport Intermédiaire}

\begin{document}

%\begin{center}
%\includegraphics[scale=0.2]{../../../Desktop/logo.jpg}
%\end{center}

% Front page.
\subfile{sections/titlepage}

% Page of contents.
\pagebreak
\tableofcontents
\pagebreak

% Introduction.
\section{Introduction}
\subfile{sections/introduction}


\pagebreak
\section{Modèle diffusif}
%\subfile{sections/section_1}
\subsection{Évolution des prix et de la richesse}
\subfile{sections/section_1_1}
\subsection{Grossissement de filtration}
\subfile{sections/section_1_2}
\subsection{Stratégies optimales}
\subfile{sections/section_1_3}
\subsection{Cas particulier : }


\pagebreak
\section{Modèle avec sauts}
\subsection{Raisonnement}
\subfile{sections/section_2_1}
\subsection{Cas particulier : }


% Bibliography.
\pagebreak
\nocite{*}
\bibliographystyle{plain}
\bibliography{projectbib}


\end{document}