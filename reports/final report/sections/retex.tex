\documentclass[../finalreport.tex]{subfiles}

\begin{document}

\par Ce mémoire est le fruit de six mois de recherche. Nous tenons en premier lieu à exprimer notre profonde reconnaissance à M. Laurent DENIS, notre tuteur, sans qui ce projet n'aurait pas été réalisable. Nous le remercions d'abord de cette opportunité de découvrir un sujet qui nous a intéressé, de développer des connaissances sur le monde du marché financier et sur l'un de ses problèmes cruciaux qu'est le délit d'initié. Nombreux sont les résultats rapportés dans ce mémoire que nous n'aurions pas pu présenter sans son aide et ses enseignements. \\

\par Ce projet nous a permis d'appliquer un certain nombre de connaissances pluridisciplinaires : interprétations économiques et simulations informatiques ont pu appuyer les résultats théoriques. Mais les outils mathématiques ont été le coeur de notre travail : nous avons réellement pu utiliser les compétences développées durant cette année d'approfondissement en Mathématiques Appliquées, et notamment celles étudiées dans le cours \emph{MAP552 - Modèles stochastique en finance}. Ce projet a en outre été l'occasion de rencontrer des outils et méthodologies qui n'ont pas été vues en cours : \emph{stratégie de consommation-placement admissible}, \emph{processus de Poisson}, \emph{exponentielle de Doléans-Dade}, \emph{méthode de grossissement de filtration}, ...\\

\par D'un point de vue scientifique, ce projet a été le premier de notre scolarité qui nous a mis au contact des publications et du monde de la recherche en Mathématiques Financières. Ce processus de lecture d'articles nous a d'abord troublé, car ceux-ci font souvent appel à des outils techniques auxquels nous ne sommes pas familiers (\emph{Calcul de Malliavin}, \emph{Formes de Dirichlet}, ...), et qui sont au dessus de notre niveau actuel. Mais nous sommes parvenus à faire abstraction de ces outils complexes et avons pu obtenir des résultats intéressants dans des cas particuliers; c'est pourquoi nous avons pris le parti dans ce mémoire de négliger les aspects "techniques" que nous ne maitrisons pas au profit d'analyses plus "concrètes" mais mieux assimilées.\\

\par Enfin, nous avons pu nous confronter au travail de recherche en binôme, avec lequel nous n'étions pas particulièrement familier; et nous pensons rétrospectivement que le travail mutuel a pu apporter une plus-value à la réalisation du projet. 

\end{document}