\documentclass[../finalreport.tex]{subfiles}

\begin{document}

\par Dans cette première partie, nous considérons un modèle de marché financier sur un espace de probabilité filtré $(\Omega, \mathcal{F}_t; t \in [0,T], \mathbb{P})$, dont les prix des actions (un actif sans risque et $d$ actifs risqués) sont régis selon l'équation : 
\begin{equation}\label{equation_prices}
\begin{cases}
S^0_t &= S_0^0 + \displaystyle \int_{0}^{t} S^0_s r_s ds \\
S_t^i &= S_0^i + \displaystyle \int_{0}^{t} S_s^i b_s^i ds +  \int_{0}^{t} S_s^i \sigma_s^i dW_s, 0\leq t \leq T, i = 1,...,d
\end{cases}
\end{equation}
où :

\begin{itemize}
\item $W$ est un mouvement brownien de dimension $d$ sur $(\Omega, \mathcal{F}_t; t \in [0,T], \mathbb{P})$ et $\mathcal{F}$ est la filtration naturelle qu'il engendre,
\item Les paramètres $b, \sigma$ et $r$ sont dans $\mathbb{R}^d, \mathbb{R}^{d\times d}, \mathbb{R}$ respectivement et sont supposés bornés sur $[0,T]$ et $\mathcal{F}$-adaptés.
\item La matrice $\sigma_t$ est inversible $dt \otimes d\mathbb{P}$-presque sûrement. On note $\eta_t = \sigma_t^{-1} \left( b_t - r_t \mathds{1} \right)$ 
\end{itemize}
\

\par L'information connue au temps $t$ par les investisseurs sur le marché est $\mathcal{F}_t$.
Nous supposons que l'initié dispose en outre, dès le début de son investissement à $t=0$, d'une information supplémentaire sous la forme d'une variable aléatoire $L \in L^1(\Omega, \mathcal{F}_T)$ sur l'espace de probabilité filtré $(\Omega, \mathcal{F}_t; t \in [0,T], \mathbb{P})$. 
Notons alors $\mathcal{Y}$ la filtration "naturelle" de l'initié obtenue par grossissement initial de $\mathcal{F}$, lui adjoignant la variable aléatoire $L : \mathcal{Y}_t = \mathcal{F}_t \vee \sigma(L), t \in [0, T]$.\\

\par A l'instant $t=0$, l'initié dispose d'un capital $X_0$. Il consomme à une vitesse $c$ (processus positif $\mathcal{Y}$-adapté vérifiant $\int_{0}^{T} c_s ds < \infty$ ), et place sur l'actif $i$ la quantité $\theta^i$. Notons $\pi_t^i = \theta^i_t S_t^i$ la somme investie sur la $i$-ième action pour $i =1, ..., d$. Sa richesse au temps $t$ s'exprime donc par : 
\begin{center}
$X_t = \displaystyle \sum_{i=0}^{d} \theta^i_t S_t^i - \int_{0}^{t} c_s ds$
\end{center}

\par Exploitons cette première équation en introduisant l'hypothèse naturelle d'autofinancement :
\begin{center}
$dX_t = \displaystyle \sum_{i=0}^{d} \theta^i_t dS_t^i - c_t dt$
\end{center}

\par Notons $R_t = (S^0_t)^{-1}$ le facteur d'actualisation, alors la formule d'Itô donne 
\mathleft
\begin{flalign*}
dX_t &= \displaystyle \sum_{i=0}^{d} \theta^i_t dS_t^i - c_t dt \\
& = \displaystyle \sum_{i=1}^{d} \theta^i_t dS_t^i + \theta^0_t dS_t^0 - c_t dt, \text{ avec } dS_t^0 = S^0_t r_t dt \\
& = \displaystyle \sum_{i=1}^{d} \theta^i_t \big(S^i_t b^i_t dt +  S^i_t \sigma^i_t dW_t \big) + \theta^0_t S^0_t r_t dt - c_t dt\\
& = \displaystyle \sum_{i=1}^{d} \pi^i_t S^i_t b^i_t dt +  \sum_{i=1}^{d} \pi^i_tS^i_t \sigma^i_t dW_t + (X_t - \sum_{i=1}^{d} \pi^i_t)r_t dt - c_t dt\\
& = (X_t r_t - c_t)dt +  \sum_{i=1}^{d}\pi^i_t(b^i_t - r_t)dt + \sum_{i=1}^{d} \pi^i_tS^i_t \sigma^i_t dW_t \\
&= (X_t r_t - c_t)dt + (\pi_t, b_t - r_t\textbf{1})dt + (\pi_t, \sigma_t dW_t) \\
\end{flalign*}

\par Nous avons encore par Itô
\begin{flalign*}
d(R_t) &= \displaystyle -\frac{dS^0_t}{(S^0_t)^2} + \frac{1}{2(S_t^0)^3} d<S^0>_t\\
& = \displaystyle -\frac{r_t}{S^0_t}dt = -r_t R_t dt
\end{flalign*}

Et
\begin{flalign*}
d(X_t R_t) &= X_t dR_t + R_t dX_t + d<X, R>_t \\
&= -X_t r_t R_t dt + R_t (X_t r_t - c_t)dt + R_t(\pi_t, b_t - r_t\textbf{1})dt + R_t (\pi_t, \sigma_t dW_t)\\
& = -R_t c_t dt + (R_t \pi_t,  b_t - r_t\textbf{1})dt + (R_t \pi_t, \sigma_t dW_t)
\end{flalign*}
Ainsi, la richesse $X$ actualisée de l'initié vérifie l'équation : 

\begin{equation}\label{equation_wealth}
\displaystyle X_tR_t + \int_{0}^{t} R_s c_s ds + \int_{0}^{t} (R_s \pi_s,  b_s - r_s\textbf{1})ds + \int_{0}^{t}(R_s \pi_s, \sigma_s dW_s)
\end{equation}
\

\end{document}