\documentclass[../finalreport.tex]{subfiles}

\begin{document}

\par Nous avons ainsi vu dans la partie précédente que le modèle diffusif permet d'étudier la stratégie de l'initié par grossissement de filtration. Nous souhaitons désormais complexifier ce modèle en lui adjoignant un processus ponctuel de type Poisson, afin de faire apparaître des "sauts" dans l'évolution des prix du marché, modélisant ainsi des situations de "catastrophes" boursières.

\par Pour cela, nous considérons un marché financier sur un espace de probabilité filtré $(\Omega, \mathcal{F}_t; t \in [0, T], \mathbb{P})$ dont les prix des actions sont dirigés par un mouvement brownien et un processus ponctuel, évoluant donc selon l'équation : 
\begin{equation}
\begin{cases}
S^0_t &= S_0^0 + \displaystyle \int_{0}^{t} S^0_s r_s ds \\
S_t^i &= S_0^i + \displaystyle \int_{0}^{t}S_s^i b_s^i ds + \int_{0}^{t}S_s^i \sigma^{ij}_t \sum_{j = 1}^{d} d(W^*, N^*)^{*}_j(s), 0 \leq t \leq T, i = 1, ..., d
\end{cases}
\end{equation}

où $W$ est un mouvement brownien de dimension $m$ sur l'espace de probabilité filtré $(\Omega^W, \mathcal{F}^W_t; t \in [0, T], \mathbb{P}^W)$, $N$ est un processus de Poisson de dimension $n$ et d'intensité $\kappa$ sur l'espace de probabilité filtré $(\Omega^N, \mathcal{F}^N_t; t \in [0, T], \mathbb{P}^N)$, avec $n + m = d$. Nous supposons ces deux processus indépendants et nous notons $(\Omega, \mathcal{F}_t; t\in [0, T], \mathbb{P}) := (\Omega^W \times \Omega^N, \mathcal{F}^W \otimes \mathcal{F}^N, \mathbb{P}^W \otimes \mathbb{P}^N)$ l'espace de probabilité filtré produit. Par la suite, nous considérerons aussi le processus de Poisson compensé $M \left( t \right) := M \left( t \right) - \int_{0}^{t} \kappa \left( s \right) ds$  \\

\par Nous conservons les notations et hypothèses du modèle précédent, de sorte que l'initié dispose d'un capital initial $X_0$, consomme toujours à une vitesse $c$, et sa stratégie est autofinançante, ce qui nous permet d'adapter le calcul de la partie précédente et d'exprimer la richesse actualisée de l'initié sous la forme de l'équation
\begin{displaymath}
X_t R_t + \int_{0}^{t} R_s c_s ds = X_0 + \int_{0}^{t} R_s\pi^*(b_s - r_s I_d)ds + \int_{0}^{t} R_s\pi^* \sigma_s d(W^*, N^*)^*(s)
\end{displaymath}


%Les procédures sont pareilles. L'initié a des informations sur le futur, représentées par la variable $L$, qui ne sont pas accessibles aux autres investisseurs sur le marché et nous notons $\mathcal{Y}$ sa filtration dite naturelle qui est $\mathcal{Y}_t = \mathcal{F}_t \vee \sigma(L), t \in [0, T]$. La méthode de grossissement de filtration, le changement de probabilité pour nous ramener à une mesure neutre au risque, et certaines hypothèses (que nous n'expliciterons pas en détaille mais se trouvent dans l'article de \textit{C. Hillairet, Comparison of insiders' optimal strategies depending on the type of side-information}) nous donneront en résultat la richesse et le portefeuille optimal de l'initié.\\

\end{document}
