<<<<<<< HEAD
\documentclass[../finalreport.tex]{subfiles}
=======
\documentclass[11pt,letterpaper]{article}
\usepackage[latin1]{inputenc}
\usepackage[french]{babel}
\usepackage{amsmath}
\usepackage{amsfonts}
\usepackage{amssymb}
\usepackage{graphicx}
\usepackage{amsthm}
\usepackage{url}
\usepackage[left=3cm,right=3cm,top=3cm,bottom=3cm]{geometry}
\usepackage{makeidx}
\makeatletter
\newcommand{\mathleft}{\@fleqntrue\@mathmargin0pt}
\newcommand{\mathcenter}{\@fleqnfalse}
\makeatother
\setlength\parindent{0pt}
\makeindex
>>>>>>> 27e2910a4b796ea8edaaf738094aebe2dc25a7b6

\begin{document}

\subsection{Description du modèle, évolution des prix et de la richesse}
Pour la deuxième partie de notre projet, nous avons étudié un modèle de marché financier qui est différent du premier en raison de la présence d'un processus ponctuel, ou un processus de Poisson qui fait de sorte que les prix comportent des sauts.\\

Nous considérons un marché financier sur un espace de probabilité filtré $(\Omega, \mathcal{F}_t; t \in [0, T], \mathbb{P})$ dont les prix des actions sont dirigés par un mouvement brownien et un processus ponctuel et évoluent selon l'équation : 

\begin{equation}
\begin{cases}S_t^i = S_0^i + \int_{0}^{t}S_s^i b_s^i ds + \int_{0}^{t}S_s^i \sigma^{ij}_t \sum_{j = 1}^{d} d(W^*, N^*)^{*}_j(s), 0 \leq t \leq T, i = 1, ..., d \\
S^0_t = \int_{0}^{t} S^0_s r_s ds \end{cases}
\end{equation}

où\\

\begin{itemize} 
\item $W$ est un mouvement brownien de dimension $m$ sur l'espace de probabilité filtré $(\Omega^W, \mathcal{F}^W_t; t \in [0, T], \mathbb{P}^W)$,
\item $N$ est un processus de Poisson de dimension $n$ sur l'espace de probabilité filtré $(\Omega^N, \mathcal{F}^N_t; t \in [0, T], \mathbb{P}^N)$,
\item $d = m+n$ et $X^*$ est le transposé de $X$,
\item $b$ et $\phi$ sont déterministes et bornés sur $[0, T]$,
<<<<<<< HEAD
\item $\sigma$ est une matrice déterministe $d \times d$,
\item $S_0$ évolue selon l'équation $dS^0_t = S^0_t r_t dt$.
=======
\item $\sigma$ est une matrice déterministe $d \times d$.
>>>>>>> 27e2910a4b796ea8edaaf738094aebe2dc25a7b6
\end{itemize}
\

Soit $(\Omega, \mathcal{F}_t; t\in [0, T], \mathbb{P}) := (\Omega^W \times \Omega^N, \mathcal{F}^W \otimes \mathcal{F}^N, \mathbb{P}^W \otimes \mathbb{P}^N)$. $W$ et $N$ sont indépendants. \\

<<<<<<< HEAD
Les procédures sont pareilles. L'initié a des informations sur le futur, représentées par la variable $L$, qui ne sont pas accessibles aux autres investisseurs sur le marché et nous notons $\mathcal{Y}$ sa filtration dite naturelle qui est $\mathcal{Y}_t = \mathcal{F}_t \vee \sigma(L), t \in [0, T]$. La méthode de grossissement de filtration, le changement de probabilité pour nous ramener à une mesure neutre au risque, et certaines hypothèses (que nous n'expliciterons pas en détaille mais se trouvent dans l'article de \textit{C. Hillairet, Comparison of insiders' optimal strategies depending on the type of side-information}) nous donneront en résultat la richesse et le portefeuille optimal de l'initié.\\
=======
Les procédures sont pareilles. L'initié a des informations sur le futur, représentées par la variable $L$, qui ne sont pas accessibles aux autres investisseurs sur le marché et nous notons $\mathcal{Y}$ sa filtration dite naturelle qui est $\mathcal{Y}_t = \mathcal{F}_t \vee \sigma(L), t \in [0, T]$. La méthode de grossissement de filtration, le changement de probabilité pour nous ramener à une mesure neutre au risque, et certaines hypothèses (que nous n'expliciterons pas en détaille mais se trouvent dans l'article de \textit{C. Hillairet, Comparison of insiders' optimal strategies depending on the type of side-information, Université Paul Sabatier, UFR MIG, Laboratoire de Statistique et Probabilités, 118 route de Narbonne, 31062 Toulouse cedex 4, France.}) nous donneront en résultat la richesse et le portefeuille optimal de l'initié.\\
>>>>>>> 27e2910a4b796ea8edaaf738094aebe2dc25a7b6

A l'instant $t = 0$, l'initié dispose d'un capital $X_0$ et consomme toujours à une vitesse $c$ qui est un processus positif $\mathcal{Y}$-adapté, vérifiant $\int_{0}^{T} c_s ds < \infty$. En notant toujours $\pi^i_t = \theta^i_t S^i_t$ la somme investie sur la i-ème action pour $i = 1,..., d$, la richesse au temps $t$ de l'initié est : \begin{equation*}
X_t = \displaystyle \sum_{i = 0}^{d} \theta^i_t S^i_t - \int_{0}^{t} c_s ds
\end{equation*}
\

Comme toujours, nous supposons que la stratégie est autofinançante, c'est-à-dire
\begin{equation*}
\displaystyle dX_t = \displaystyle \sum_{i = 0}^{d} \theta^i_t dS^i_t - c_t dt.
\end{equation*}
\

Avec le facteur d'actualisation $R_t = (S^0_t)^{-1}$, sa richesse actualisée vérifie l'équation donnée par la formule d'Itô suivante :
\mathleft
\begin{flalign*}
\displaystyle dX_t &= \sum_{i=0}^{d}\theta^i_t dS^i_t - c_t dt\\
&= \displaystyle \sum_{i=1}^{d}\theta^i_t dS^i_t + \theta^0_t dS^0_t - c_t dt\\
&= \displaystyle \sum_{i=1}^{d}\theta^i_t\Big(S^i_t b^i_t dt + S^i_t \sum_{j=1}^{d}\sigma^{ij}_td(W^*, N^*)^*_j (t)\Big) + \theta^0_t S^0_t r_tdt - c_t dt\\
&= \displaystyle (X_t r_t - c_t) dt + \sum_{i=1}^{d} \pi^i_t (b^i_t - r_t) dt + \sum_{i=1}^{d} \pi^i_t \sum_{j=1}^{d}\sigma^{ij}_td(W^*, N^*)^*_j (t) \\
&= (X_t r_t - c_t) dt + \pi^*(b_t - r_t I_d)dt + \pi^*\sigma_t d(W^*, N^*)^*(t) ,
\end{flalign*}
<<<<<<< HEAD
avec $\pi = (\pi^1_t, ..., \pi^d_t)^{*}$ et $ I_d = (1, ..., 1)^{*} \in \mathbb{R}^d $.\\
=======
où $\pi = (\pi^1_t, ..., \pi^d_t)^*$ et $ I_d = (1, ..., 1)^* \in \mathbb{R}^d$.\\
>>>>>>> 27e2910a4b796ea8edaaf738094aebe2dc25a7b6

Avec $d(R_t) = d\big((S^0_t)^{-1}\big) = -r_t R_t dt$, la formule d'Itô nous donne encore 
\begin{flalign*}
\displaystyle d(X_tR_t) &= X_t dR_t + R_t dX_t + d<X, R>_t \\
&= \displaystyle -X_t r_t R_t dt + R_t \big(  (X_t r_t- c_t) dt + \pi^*(b_t - r_t I_d)dt + \pi^*\sigma_t d(W^*, N^*)^*(t) \big)\\
&= \displaystyle -R_t c_t dt + R_t\pi^*(b_t - r_t I_d)dt + R_t\pi^*\sigma_t d(W^*, N^*)^*(t)
\end{flalign*}

$\displaystyle \Rightarrow X_t R_t + \int_{0}^{t} R_s c_s ds = X_0 + \int_{0}^{t} R_s\pi^*(b_s - r_s I_d)ds + \int_{0}^{t} R_s\pi^* \sigma_s d(W^*, N^*)^*(s)$\\

En notant : \\
\begin{itemize}
\item $\Theta_t := $ les $m$ premières lignes de $(\sigma_t)^{-1}(b_t - r_t I_d)$.\
\item $q$ un processus de dimension $n$, dont les composants sont supposés positifs, tel que $q_t. \kappa_t :=$ les n dernière lignes de $-(\sigma_t)^{-1}(b_t - r_t I_d)$,
\end{itemize}
\

<<<<<<< HEAD
avec $q_t. \kappa_t$ définit le vecteur dont les composants sont $(q_t.\kappa_t)_i = q^i_t\kappa^i_t, i = 1,...,n$, nous définissons : 
=======
où $q_t. \kappa_t$ définit le vecteur dont les composants sont $(q_t.\kappa_t)_i = q^i_t\kappa^i_t, i = 1,...,n$, nous définissons : 
>>>>>>> 27e2910a4b796ea8edaaf738094aebe2dc25a7b6
\begin{itemize}
\item $\displaystyle \widehat{W}_t : = W_t + \int_{0}^{t}\Theta_s ds$\
\item $\displaystyle \widehat{M}_t : = N_t - \int_{0}^{t} q_s. \kappa_s ds$\
\item $\widehat{S} : = (\widehat{W}^*, \widehat{N}^*)^*$
\end{itemize}
\

Donc notre équation de richesse actualisée en fonction de $\Theta_t$ et $q_t.\kappa_t$ devient : 
\begin{flalign*}
X_t R_t + \int_{0}^{t} R_s c_s ds &= X_0 + \int_{0}^{t} R_s\pi^*(b_s - r_s I_d)ds + \int_{0}^{t} R_s\pi^* \sigma_t d(W^*, N^*)^*(s)\\
&= X_0 + \int_{0}^{t} R_s\pi^*(b_s - r_s I_d)ds + \int_{0}^{t} R_s\pi^* \sigma_s d(\widehat{W}^*, \widehat{N}^*)^*(s)\\
 & -  \int_{0}^{t} R_s\pi^* \sigma_s \Theta_s ds +  \int_{0}^{t} R_s\pi^* \sigma_t q_s.\kappa_s ds\\
& = X_0 + \int_{0}^{t} R_s\pi^*(b_s - r_s I_d)ds + \int_{0}^{t} R_s\pi^* \sigma_s d(\widehat{W}^*, \widehat{N}^*)^*(s)\\
& -  \int_{0}^{t} R_s\pi^* \sigma_s (\sigma_s)^{-1}  \big(\Theta_s I_{1,m} - q_s. k_s I_{2,n}) \\
& = X_0 + \int_{0}^{t} R_s\pi^*(b_s - r_s I_d)ds + \int_{0}^{t} R_s\pi^* \sigma_s d(\widehat{W}^*, \widehat{N}^*)^*(s) - \int_{0}^{t} R_s\pi^*(b_s - r_s I_d)ds\\
& = X_0 + \int_{0}^{t} R_s\pi^* \sigma_s d(\widehat{W}^*, \widehat{N}^*)^*(s)\\
&= X_0 + \int_{0}^{t} R_s\pi^* \sigma_s d\widehat{S} (s)
\end{flalign*}
\
<<<<<<< HEAD
ou $I_{1, m} = ( \underbrace{1,...,1}_{m}, \underbrace{0,...,0}_n)^*, I_{2, n} = (\underbrace{0,...,0}_m, \underbrace{1,...,1}_{n})^*$\\
=======
où $I_{1, m} = ( \underbrace{1,...,1}_{m}, \underbrace{0,...,0}_n)^*, I_{2, n} = (\underbrace{0,...,0}_m, \underbrace{1,...,1}_{n})^*$\\
>>>>>>> 27e2910a4b796ea8edaaf738094aebe2dc25a7b6

\subsection{Raisonnement}
Comme dans la première partie, l'initié cherche à optimiser sa stratégie de manière suivante : 
\begin{flalign*}
J : &\mathcal{A} \rightarrow \mathbb{R}\\
&(\pi, c) \mapsto J(X_0, \pi, c) = \mathbb{E}_{\mathbb{P}} \Big[ \displaystyle \int_{0}^{A} U_1(c_t)dt + U_2(X_A^{\pi, c})\Big |\mathcal{Y}_0\Big]
\end{flalign*}
<<<<<<< HEAD
ou $\mathcal{A}, (U_1, U_2)$ et $A$ sont les mêmes notations que celles dans le marché diffusif.\\
=======
où $\mathcal{A}, (U_1, U_2)$ et $A$ sont les mêmes notations que celles dans le marché diffusif.\\
>>>>>>> 27e2910a4b796ea8edaaf738094aebe2dc25a7b6

Les problèmes auxquels nous faisons face ici viennent toujours du fait que sur l'espace $(\Omega, \mathcal{Y}, \mathbb{P})$, les processus $W^*$ et $N^*$ ne sont plus des semi-martingales. Pour cela, nous allons : \\
\begin{itemize}
\item D'abord trouver une mesure de probabilité $\mathbb{Q}$ équivalente à $\mathbb{P}$ sous laquelle $\mathcal{F}_t$ et $\sigma(L)$ sont indépendants $\forall t \in [0, T[$.\
\item Ensuite utiliser la méthode de grossissement de filtration pour construire un mouvement brownien et un processus de Poisson sur l'espace de probabilité filtré $(\Omega, \mathcal{Y}, \mathbb{P})$ à l'aide de notre mesure de probabilité $\mathbb{Q}$.
\item Nous ramener finalement à une mesure risque-neutre $\mathbb{Q}_1$ par un dernier changement de probabilité. 
\end{itemize}
\

\subsubsection{Changement de probabilité}
<<<<<<< HEAD
\textbf{Proposition 3.1.} \textit{Il existe une mesure de probabilité $\mathbb{Q}$ qui est équivalente à $\mathbb{P}$ sur $\mathcal{F}_T$ et sous cette probabilité, $\mathcal{F}_t$ et $\sigma(L)$ sont indépendants $\forall t \in [0, T[$}.\\
=======
\textbf{Proposition 3.1.} \textit{Il existe une mesure de probabilit� $\mathbb{Q}$ qui est �quivalente � $\mathbb{P}$ sur $\mathcal{F}_T$ et sous cette probabilit�, $\mathcal{F}_t$ et $\sigma(L)$ sont ind�pendants $\forall t \in [0, T[$}.\\
>>>>>>> 27e2910a4b796ea8edaaf738094aebe2dc25a7b6

\subsubsection{Grossissement de filtration}
Introduisons la densité : 
\mathcenter
\begin{equation*}
\displaystyle Z_t : = \mathbb{E}_{\mathbb{Q}}\Big[\frac{d\mathbb{P}}{d\mathbb{Q}}|Y_t\Big]
\end{equation*}
\
<<<<<<< HEAD
qui satisfait l'équation $\displaystyle dZ_t = Z_t\Big(\rho^*_1 (t) dW_t + (\rho^*_2 (t) - I_n)^* dM_t\Big)$, ou $\rho_1$ et $\rho_2$ sont des processus $\mathcal{Y}_T$-prévisibles.\\
=======
qui satisfait l'équation $\displaystyle dZ_t = Z_t\Big(\rho^*_1 (t) dW_t + (\rho^*_2 (t) - I_n)^* dM_t\Big)$, où $\rho_1$ et $\rho_2$ sont des processus $\mathcal{Y}_T$-prévisibles.\\
>>>>>>> 27e2910a4b796ea8edaaf738094aebe2dc25a7b6

En corollaire, les processus \\

\begin{itemize}
\item $\displaystyle \widetilde{W}_t := W_t - \int_{0}^{t}\ \rho_1(s) ds$\
\item $\displaystyle \widetilde{M}_t := N_t - \int_{0}^{t}\kappa.\rho_2 (s) ds$
\end{itemize}
\ 

sont un mouvement brownien et un processus de Poisson de l'intensité $(\kappa.\rho_2)$ respectivement sur $(\Omega, \mathcal{Y}, \mathbb{P})$.\\

\subsubsection{Probabilité risque-neutre}
Notons
\begin{equation*}
Y_t := \varepsilon \Big(\displaystyle \int_{0}^{t} \big(-(\Theta + \rho_1(s))^* d\widetilde{W}_s + (\frac{q}{\rho_2(s)} - I_n )^* d\widetilde{M}_s\big)\Big), 
\end{equation*}
<<<<<<< HEAD
ou $\varepsilon$ est l'exponentielle de Doléans-Dade. \\
=======
où $\varepsilon$ est l'exponentielle de Doléans-Dade. \\
>>>>>>> 27e2910a4b796ea8edaaf738094aebe2dc25a7b6

Alors $Y$ est une $(\mathcal{Y}_T, \mathbb{P})$-martingale locale positive, et $\mathbb{Q}_1:= Y \mathbb{P}$ définit la probabilité risque-neutre pour l'initié. Les processus $\widetilde{W}$ et $\widetilde{M}$ sont respectivement un mouvement brownien et un processus de Poisson de l'intensité $(q.\kappa)$ sur $(\Omega, \mathcal{Y}, \mathbb{Q}_1)$.\\

Par calculs, nous avons $\displaystyle d(Y^{-1}_t) = Y^{-1}_t l^*_t d\widehat{S}_t$, avec $\displaystyle l^*_t := \big((\Theta + \rho_1)^*, (\frac{\rho_2}{q} - I_n)^*\big)$.\\

\subsection{Calcul des stratégies optimales}
Le problème d'optimisation $ \mathbb{E}_{\mathbb{P}} \Big[ \displaystyle \int_{0}^{A} U_1(c_t)dt + U_2(X_A^{\pi, c})\Big|\mathcal{Y}_0\Big]$ sur toutes les $\mathcal{Y}$-stratégies admissibles que nous cherchons depuis le début ont pour solutions : \\
 
$A<T, \forall t \in [0, A], \quad \begin{cases}
\displaystyle R_t \widehat{c}_t = \frac{X_0}{A+1} \frac{1}{Y}(t)\\
\displaystyle R_t \widehat{X}_t = \frac{X_0(A+1-t)}{A+1}\frac{1}{Y}(t)\\
\displaystyle \widehat{\pi}_t = (\sigma^*_t)^{-1}\frac{Y(t)}{Y(t^{-1})}\widehat{X}_t l_t
\end{cases}$



\end{document}
