\documentclass[11pt,letterpaper]{article}
\usepackage[latin1]{inputenc}
\usepackage[french]{babel}
\usepackage{amsmath}
\usepackage{amsfonts}
\usepackage{amssymb}
\usepackage{graphicx}
\usepackage{amsthm}
\usepackage{url}
\usepackage[left=3cm,right=3cm,top=3cm,bottom=3cm]{geometry}
\usepackage{makeidx}
\makeatletter
\newcommand{\mathleft}{\@fleqntrue\@mathmargin0pt}
\newcommand{\mathcenter}{\@fleqnfalse}
\makeatother
\setlength\parindent{0pt}
\makeindex
\begin{document}

Pour la deuxi�me partie de notre projet, nous avons �tudi� un mod�le de march� financier qui est diff�rent du premier en raison de la pr�sence d'un processus ponctuel, ou un processus de Poisson qui fait de sorte que les prix comportent des sauts.\\

Nous consid�rons un march� financier sur un espace de probabilit� filtr� $(\Omega, \mathcal{F}_t; t \in [0, T], \mathbb{P})$ dont les prix des actions sont dirig�s par un mouvement brownien et un processus ponctuel et �voluent selon l'�quation : 

\begin{equation}
S_t^i = S_0^i + \displaystyle \int_{0}^{t}S_s^i b_s^i ds + \int_{0}^{t}S_s^i \sum_{j = 1}^{d} d(W^*, N^*)^{*}_j(s), 0 \leq t \leq T, i = 1, ..., d
\end{equation}

o�\\

\begin{itemize} 
\item $W$ est un mouvement brownien de dimension $m$ sur l'espace de probabilit� filtr� $(\Omega^W, \mathcal{F}^W_t; t \in [0, T], \mathbb{P}^W)$,
\item $N$ est un processus de Poisson de dimension $n$ sur l'espace de probabilit� filtr� $(\Omega^N, \mathcal{F}^N_t; t \in [0, T], \mathbb{P}^N)$,
\item $d = m+n$ et $X^*$ est le transpos� de $X$,
\item $b$ et $\phi$ sont d�terministes et born�s sur $[0, T]$,
\item $\sigma$ est une matrice d�terministe $d \times d$,
\item $S_0$ �volue selon l'�quation $dS^0_t = S^0_t r_t dt$.
\end{itemize}
\

Soit $(\Omega, \mathcal{F}_t; t\in [0, T], \mathbb{P}) := (\Omega^W \times \Omega^N, \mathcal{F}^W \otimes \mathcal{F}^N, \mathbb{P}^W \otimes \mathbb{P}^N)$. $W$ et $N$ sont ind�pendants. \\

Les proc�dures sont pareilles. L'initi� a des informations sur le futur, repr�sent�es par la variable $L$, qui ne sont pas accessibles aux autres investisseurs sur le march� et nous notons $\mathcal{Y}$ sa filtration dite naturelle qui est $\mathcal{Y}_t = \mathcal{F}_t \vee \sigma(L), t \in [0, T]$. La m�thode de grossissement de filtration, le changement de probabilit� pour nous ramener � une mesure neutre au risque, et certaines hypoth�ses (que nous n'expliciterons pas en d�taille mais se trouvent dans l'article de \itshape{C. Hillairet, Comparison of insiders' optimal strategies depending on the type of side-information, Universit� Paul Sabatier, UFR MIG, Laboratoire de Statistique et Probabilit�s, 118 route de Narbonne, 31062 Toulouse cedex 4, France.}) nous donneront en r�sultat la richesse et le portefeuille optimal de l'initi�.
\end{document}
