\documentclass{beamer}
\usepackage[utf8]{inputenc}
\usepackage[french]{babel}
\usepackage{amsmath}
\usepackage{amsfonts}
\usepackage{amssymb}
\usepackage{manfnt}
\usepackage{graphicx}
\usepackage{amsthm}
\usepackage{url}
\usepackage{makeidx}
\usepackage{mathtools}
\usepackage{dsfont}
\usepackage{hyperref}
\usepackage{float}
\usepackage{makeidx}
\makeatletter
\newcommand{\mathleft}{\@fleqntrue\@mathmargin0pt}
\newcommand{\mathcenter}{\@fleqnfalse}
\makeatother
\setlength\parindent{0pt}
\makeindex

\usetheme{Berlin}
\title{Modélisation et détection de délit d'initié}
\author{BROUX Lucas, HEANG Kitiyavirayuth}
\date{20 mars 2018}

\AtBeginSection[]{
  \begin{frame}
  \vfill
  \centering
  \begin{beamercolorbox}[sep=8pt,center,shadow=true,rounded=true]{title}
    \usebeamerfont{title}\insertsectionhead\par%
  \end{beamercolorbox}
  \vfill
  \end{frame}
}


% Prévoir 30 minutes de passage, 10 minutes de questions.

\begin{document}

%Title page
\begin{frame}
\titlepage
\end{frame}

\section*{Introduction}
\subsection*{Projet}
\begin{frame}
\frametitle{Présentation du projet}
\begin{itemize}
\item \textbf{Modélisation et détection de délit d'initié :}
\par Que se passe t'il lorsqu'un agent dispose d'une information confidentielle sur l'évolution future du marché ?
\item Objectifs :
	\begin{itemize}
	\item Analyser le gain de l'initié par rapport à un non-initié.
	\item Simuler la richesse de l'initié et du non-initié.
	\end{itemize}
\end{itemize}
\end{frame}

\subsection*{Choix du sujet}
\begin{frame}
\frametitle{Choix du sujet}
\begin{itemize}
\item Problématique concrète.
\item Aspect théorique : notions profondes et techniques.
\item Étude de cas particuliers possible à notre niveau.
\end{itemize}
\end{frame}

%Table of contents
\section*{Plan}
\begin{frame}
\frametitle{Plan de la présentation}
\tableofcontents
\end{frame}

%Sections MODELE DIFFUSIF 
\section{Modèle diffusif - cas particulier}

%SUBSECTION DESCRIPTIF DU MODELE 
\subsection{Description du modèle}

%EVOLUTION DES PRIX
\begin{frame}
\frametitle{Marché}
On considère 2 actions risquées sur le marché financier sur l'espace de probabilité $(\Omega, \mathcal{F}_t; t \in[0, T], \mathbb{P})$, dont les prix évoluent selon l'équation : 
\begin{equation*}
\begin{cases} 
S^0_t = S^0_0 +  \int_{0}^{t} S^0_s r_s ds \\
S^1_t = S^1_0 + \int_{0}^{t} S^1_s b^1_s ds + \int_{0}^{t} S^1_s \sigma^1_s dW_s, \quad 0 \leq t \leq T\\
S^2_t = S^2_0 + \int_{0}^{t} S^2_s b^2_s ds + \int_{0}^{t} S^2_s \sigma^2_s dW_s, \quad 0 \leq t \leq T
\end{cases}
\end{equation*}
\begin{itemize}
\item $W$ est un mouvement brownien à 2 dimensions dont $\mathcal{F}$ est la filtration naturelle.\\
\item Pour simplifier, on suppose que $b, r$ et $\sigma$ sont constants.
\end{itemize}
\end{frame}

\begin{frame}
\frametitle{Marché}
Les prix peuvent donc s'expliciter : 
\begin{displaymath}
\begin{cases}
S_t^0 &= S_0^0 e^{ r t } \\
S_t^1 &= S_0^1 e^{ \left( b_1 - \frac{1}{2} ||\sigma_1||^2 \right) t + \left( \sigma_1, W \left( t \right) \right) }  \\
S_t^2 &= S_0^2 e^{ \left( b_2 - \frac{1}{2} ||\sigma_2||^2 \right) t + \left( \sigma_2, W \left( t \right) \right) }
\end{cases}
\end{displaymath}

\end{frame}

%INITIE
\begin{frame}
\frametitle{Initié}
\begin{itemize}
\item On suppose qu'à $t= 0$, l'initié dispose a une information sur le futur, $L := \ln(S^1_T) - \ln(S^2_T)$, dont les autres investisseurs sur le marché ne disposent pas.  \\
\item Sa filtration naturelle est donc $\mathcal{Y}_t := \mathcal{F}_t \vee \sigma(L)$.\\
\item Il dispose d'un capital $X_0$ à $t=0$, consomme à une vitesse $c$, et il place la quantité $\theta^i$ sur l'actif $i$.\\
\item $\pi_t^i = \theta^i_t S^i_t$ : la somme investie sur le $i$-ième l'actif, $i= \{1, 2\}$.
\end{itemize}
\end{frame}

%HYPOTHESE D'AUTOFINACEMENT
\begin{frame}
\frametitle{Hypothèse d'autofinancement}
\begin{itemize}

\item Sa richesse au temps $t$ s'exprime donc : 
\begin{equation*}
X_t = \displaystyle \theta^0_t S^0_t  + \theta^1_t S^1_t  + \theta^2_t S^2_t - \int_{0}^{t} c_s ds
\end{equation*}

\item Nous supposons que son portefeuille est autofinançant : 
\begin{equation*}
dX_t = \displaystyle  \theta^0_t dS^0_t + \theta^1_t dS^1_t + \theta^2_t dS^2_t - c_t dt
\end{equation*}

\item En notant $R_t = (S^0_t)^{-1}$ le facteur d'actualisation, on obtient : 
\begin{equation*}
X_t R_t + \int_{0}^{t} R_s c_s ds = \int_{0}^{t} (R_s \pi_s, b_s - r_s \textbf{1})ds + \int_{0}^{t} (R_s \pi_s, \sigma_s dW_s)
\end{equation*}
\end{itemize}
\end{frame}


\subsection{Stratégie optimale}
\begin{frame}
\frametitle{Stratégie optimale}
\par L'initié cherche à optimiser son "utilité" :
\small
\begin{align*}
      J \colon &\underbrace{\mathcal{A}}_{\text{Stratégies admissibles}} \to \mathbb{R} \\
       &\left( \pi, c \right) \xmapsto{\phantom{\mathcal{A}}} J \left( X_0, \pi, c \right) := \mathbb{E} \left[ \int_0^A \log  \left( c_t \right) dt + \log \left( \underbrace{X_A^{\pi, c}}_{\text{Richesse au temps } A} \right) \left\vert\right. \mathcal{Y}_0 \right]
\end{align*}

\end{frame}

\begin{frame}
\frametitle{Stratégie optimale}

\begin{itemize}

\item où
\begin{displaymath}
\mathcal{A} = \left\lbrace \left( \pi, c \right), 
\begin{cases} 
\pi \enskip \mathcal{Y}-\text{prévisible}, \\
c > 0 \enskip \mathcal{Y}-\text{adapté}, \\ 
\int_0^T c_s ds < + \infty \enskip \text{et} \enskip \sigma^* \pi \in L^2 \left[ 0; T \right] \enskip \mathbb{P}-p.s., \\
X^{\pi, c} \geq 0 \enskip dt \otimes d \mathbb{P}-p.s. 
\end{cases}
\right\rbrace
\end{displaymath} 
\item $A < T$ : \emph{Temps final}.
\end{itemize}
\par Peut-on caractériser $\mathcal{A}$ sous une forme exploitable ?
\end{frame}


\subsection{Résolution du problème d'optimisation}
\begin{frame}
\frametitle{Raisonnement}
\end{frame}

\begin{frame}
\frametitle{Changement de probabilités}
\end{frame}

\begin{frame}
\frametitle{Grossissement de filtration}
\end{frame}

\begin{frame}
\frametitle{Changement de probabilité}
\end{frame}

\begin{frame}
\frametitle{Caractérisation de $\mathcal{A}$}
\end{frame}

\begin{frame}
\frametitle{Résolution du problème d'optimisation}
\end{frame}

\subsection{Analyse du gain de l'initié}

\begin{frame}
\frametitle{Forme explicite du gain}
\par On a :
\begin{displaymath}
Y \left( t \right) = 
\begin{cases}
e^{- \left( \eta, W_{t} \right)-\frac{1}{2} t {\| \eta \|}^{2}} =: Y_0 \left( t \right) & \text{pour le non initié} \\
e^{- \int_{0}^{t} \left( l_{s} + \eta, dW_{s} - l_s ds \right) - \frac{1}{2} \int_{0}^{t} {\| l_{s} + \eta \|}^{2} ds} & \text{pour l'initié}
\end{cases}
\end{displaymath}

\par On s'intéresse au gain de l'initié par rapport au non-initié : 

\begin{displaymath}
	\begin{split}
	Z \left( t \right) &:= \frac{Y_0 \left( t \right)}{Y \left( t \right)} = \frac{ e^{- \left( \eta, W_{t} \right)-\frac{1}{2} t {\| \eta \|}^{2}}}{e^{- \int_{0}^{t} \left( l_{s} + \eta, dW_{s} - l_s ds \right) - \frac{1}{2} \int_{0}^{t} {\| l_{s} + \eta \|}^{2} ds}} \\
	&= e^{ \int_{0}^{t} \left[ \left( l_s, dW_{s} \right) ds \right] - \frac{1}{2} \int_{0}^{t} {\| l_{s} \|}^{2} ds}
	\end{split}
\end{displaymath}
\end{frame}
 
\begin{frame}
\frametitle{Forme explicite du gain}
\par Donc
\begin{displaymath}
d \log \left( Z \left( t \right) \right) = \underbrace{\left(l_s, d W_s \right)}_{ = \frac{d p \left( t, L \right)}{p \left( t, L \right)}} - \frac{1}{2} {\| l_{s} \|}^{2} ds = d p \left( t, L \right)
\end{displaymath}

\par Donc $\frac{Z \left( t \right)}{ p \left( t, L \right) }$ est constant :

\begin{displaymath}
	\begin{split}
	Z \left( t \right) &= \frac{Z \left( 0 \right)}{ p \left( 0, L \right) } p \left( t, L \right)
	\end{split}
\end{displaymath}
\par Interprétation : pour tout $t$, le gain proportionnel de l'initié correspond - à une constante initiale près -  à la densité conditionnelle de la variable $L$ sachant $\mathcal{F}_t$, prise en $x = L$.
\end{frame}
 
\begin{frame}
\frametitle{Forme explicite du gain}
\par On obtient donc :
\begin{displaymath}
	\begin{split}
		\boxed{
	Z \left( t \right) = \frac{\sqrt{T}}{\sqrt{T - t}} e^{\frac{- \left( \gamma, W \left( T \right) - W \left( t \right) \right)^2}{2 ||\gamma||^2 \left( T - t \right)} + \frac{\left( \gamma, W \left( T \right) \right)^2}{2 ||\gamma||^2 T }}
		}
	\end{split}
\end{displaymath}
\end{frame}

\begin{frame}
\frametitle{Analyse du gain}
\par En remplaçant $ \left( \gamma, W \left( T \right) - W \left( t \right) \right)^2 $ par son espérance $||\gamma||^2 \left( T - t \right)  $,

\begin{displaymath}
	Z \left( t \right) \simeq \frac{\sqrt{T}}{\sqrt{T - t}}
\end{displaymath}

\par \textdbend Ce raisonnement est heuristique. Pour calculer $\mathbb{E} \left[ Z \left( t \right) \right]$, il faudrait utiliser le théorème de transfert.

\end{frame}

\begin{frame}
\frametitle{Visualisation}
\par Moyenne de 10000 simulations de $Z$ (\emph{simulations\_averages.py}) : 

\end{frame}


\subsection{Simulations}

\begin{frame}
\frametitle{Simulations}
\par Nous avons implémenté numériquement les formules (\emph{repository Github} du projet : \url{https://github.com/lucas-broux/Projet-3A}).
\begin{figure}[H]
  \centering
    \includegraphics[width=0.7\textwidth]{images/github.png}
  \caption{}
\end{figure}

\end{frame}



\section{Conclusion}

\begin{frame}
\frametitle{Conclusion}
\begin{itemize}
\item Il est possible d'exprimer et d'étudier le gain de l'initié dans des cas plus ou moins particuliers.
\item Des théorèmes généraux mais techniques assurent que - sous hypothèses - le raisonnement reste vrai.
\item Simulations numériques possibles dans certains cas, mais rendues difficiles dans d'autres.
\end{itemize}
\end{frame}

\section{Retour d'expérience}

\begin{frame}
\frametitle{Retour d'expérience}
\begin{itemize}
\item Travail de lecture d'article :
	\begin{itemize}
	\item Identifier les passages trop techniques.
	\item Étudier en détail les cas particuliers.
	\end{itemize}
\item Travail en binôme.
\end{itemize}
\end{frame}


\begin{frame}
\Huge \center Merci !
\huge\center Questions ?
\end{frame}


\begin{frame}
\end{frame} % to enforce entries in the table of contents


\end{document}