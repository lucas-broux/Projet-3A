\documentclass{beamer}
\usepackage[utf8]{inputenc}
\usepackage[french]{babel}
\usepackage{amsmath}
\usepackage{amsfonts}
\usepackage{amssymb}
\usepackage{graphicx}
\usepackage{amsthm}
\usepackage{url}
\usepackage{makeidx}
\makeatletter
\newcommand{\mathleft}{\@fleqntrue\@mathmargin0pt}
\newcommand{\mathcenter}{\@fleqnfalse}
\makeatother
\setlength\parindent{0pt}
\makeindex

\usetheme{Berlin}
\title{Modélisation et détection de délit d'initié}
\author{BROUX Lucas, HEANG Kitiyavirayuth}
\date{20 mars 2018}

\AtBeginSection[]{
  \begin{frame}
  \vfill
  \centering
  \begin{beamercolorbox}[sep=8pt,center,shadow=true,rounded=true]{title}
    \usebeamerfont{title}\insertsectionhead\par%
  \end{beamercolorbox}
  \vfill
  \end{frame}
}


% Prévoir 30 minutes de passage, 10 minutes de questions.

\begin{document}

%Title page
\begin{frame}
\titlepage
\end{frame}

\section{Introduction}
\subsection{Projet}
\begin{frame}
\frametitle{Présentation du projet}
\begin{itemize}
\item \textbf{Modélisation et détection de délit d'initié :}
\par Que se passe t'il lorsqu'un agent dispose d'une information confidentielle sur l'évolution future du marché ?
\item Objectifs :
	\begin{itemize}
	\item Analyser le gain de l'initié par rapport à un non-initié.
	\item Simuler les stratégies de l'initié et du non-initié.
	\end{itemize}
\end{itemize}
\end{frame}

\subsection{Choix du sujet}
\begin{frame}
\frametitle{Choix du sujet}
\begin{itemize}
\item Problématique concrète.
\item Aspect théorique : notions profondes et techniques.
\item Étude de cas particuliers possible à notre niveau.
\end{itemize}
\end{frame}

%Table of contents
\section*{Plan}
\begin{frame}
\frametitle{Plan de la présentation}
\tableofcontents
\end{frame}

%Sections MODELE DIFFUSIF 
\section{Modèle diffusif - cas particulier}

%SUBSECTION DESCRIPTIF DU MODELE 
\subsection{Description du modèle}

%EVOLUTION DES PRIX
\begin{frame}
\frametitle{évolution des prix}
On considère 2 actions risquées sur le marché financier sur l'espace de probabilité $(\Omega, \mathcal{F}_t; t \in[0, T], \mathbb{P})$, dont les prix évoluent selon l'équation : 
\begin{equation*}
\begin{cases} S^i_t = S^0_t + \int_{0}^{t} S^s_t b^s_t dt + \int_{0}^{t} S^i_s \sigma^i_s dW_s, \quad 0 \leq t \leq T, \quad i = \{1, 2\}\\
S^0_t = S^0_0 +  \int_{0}^{t} S^0_s r_s ds \end{cases}
\end{equation*}
\begin{itemize}
\item $W$ est un mouvement brownien de 2 dimension dont $\mathcal{F}$ est la filtration naturelle.\\
\item Pour simplifier, on suppose que $b, r$ et $\sigma$ sont constants.
\end{itemize}
\end{frame}

%INITIE
\begin{frame}
\frametitle{Initié}
\begin{itemize}
\item On suppose qu'à $t= 0$, l'initié dispose a une information sur le futur, $L := \ln(S^1_T) - \ln(S^2_T)$, dont les autres investisseurs sur le marché ne disposent pas.  \\
\item Sa filtration naturelle est donc $\mathcal{Y}_t := \mathcal{F}_t \vee \sigma(L)$.\\
\item Il dispose d'un capital $X_0$ à $t=0$, consomme à une vitesse $c$, et il place la quantité $\theta^i$ sur l'actif $i$.\\
\item $\pi_t^i = \theta^i_t S^i_t$ : la somme investie sur le $i$-ième l'actif, $i= \{1, 2\}$.
\end{itemize}
\end{frame}

%HYPOTHESE D'AUTOFINACEMENT
\begin{frame}
\frametitle{Hypothèse d'autofinancement}
\begin{itemize}

\item Sa richesse au temps $t$ s'exprime donc : 
\begin{equation*}
X_t = \displaystyle \sum_{i=0}^{2} \theta^i_t S^i_t - \int_{0}^{t} c_s ds
\end{equation*}

\item Nous supposons que son portefeuille est autofinançant : 
\begin{equation*}
dX_t = \displaystyle \sum_{i=0}^{2} \theta^i_t dS^i_t - c_t dt
\end{equation*}

\item En notant $R_t = (S^0_t)^{-1}$ le facteur d'actualisation, on obtient : 
\begin{equation*}
X_t R_t + \int_{0}^{t} R_s c_s ds = \int_{0}^{t} (R_s \pi_s, b_s - r_s \textbf{1})ds + \int_{0}^{t} (R_s \pi_s, \sigma_s dW_s)
\end{equation*}
\end{itemize}
\end{frame}



\subsection{Stratégie optimale}

\subsection{Résolution du problème d'optimisation}

\subsection{Analyse du gain de l'initié}

\subsection{Simulations}



\section{Conclusion}

\section{Retour d'expérience}

\end{document}